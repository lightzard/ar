\documentclass[12pt]{article}
\usepackage{a4wide}
\usepackage{latexsym}
\usepackage{amssymb}
\usepackage{epic}
\usepackage{graphicx}
\usepackage{amsmath}
%\pagestyle{empty}
\newcommand{\tr}{\mbox{\sf true}}
\newcommand{\fa}{\mbox{\sf false}}
\newcommand{\bimp}{\leftrightarrow}


\begin{document}

\section*{Practical
Assignment Automated Reasoning 2IW15 }

\begin{center}
Irfan Nur Afif \\
1035476\\
email: {\tt irfan.nur.afif@.student.tue.nl}
\end{center}

\vspace{8mm}

\subsection*{Problem: Pallets delivery}

\begin{enumerate}
	\item Investigate what is the maximum number of pallets of prittles that can be delivered,
	and show how for that number all pallets may be divided over the eight trucks.
	\item Do the same, with the extra information that prittles and crottles are an explosive combination: they are not allowed to be put in the same truck.
\end{enumerate}
 

\vspace{8mm}

\subsection*{Solution:}

We introduce five sets of variables for every type of pallets: $a$ for nuzzles, $b$ for prittles,  $c$ for skipples, $d$ for crottles, $e$ for dupples. For every variable $k \in\{a,b,c,d,e\}$, we divide it again into $k_i, i=1, \cdots 8$ (according to the number of trucks) representing the number of pallets $k$ in truck $i$.

For a truck, it has maximum capacity of 8 pallets. This is expressed by the formula
\[a_i + b_i + c_i + d_i + e_i \leq 8  \text{, for }  i=1,2,\cdots 8 \]

Each pallets $k$ has a weight $w_k$ kg. Each truck also has a maximum weight of 8,000 kg. This is expressed by the formula:
\[w_a a_i + w_b b_i + w_c c_i + w_d d_i + w_e e_i \leq 8000  \text{, for }  i=1,2,\cdots 8 \]
where $w_a=800$ (corresponds to nuzzles weight in kg), $w_b=1100$, $w_c=1000$, $w_d=2500$, $w_3=200$. For each pallets, there are a number of items that needs to be delivered by trucks. This is represented by the formula
\[ \sum_{i=1}^{8}{a_i} \leq p_a \wedge
 \sum_{i=1}^{8}{b_i} \leq p_b \wedge 
 \sum_{i=1}^{8}{c_i} \leq p_c \wedge
 \sum_{i=1}^{8}{d_i} \leq p_d \wedge
 \sum_{i=1}^{8}{e_i} \leq p_e\]
 where $p_a=4$ (corresponds to number of nuzzles), $p_b$ is unknown, $p_c=8$, $p_d=10$, $p_e=12$. There are also two additional constraints. The first is that there are only three trucks that can carry the skipples. The way we represents this is by putting the first three trucks as the trucks that has cooling facility (thus it can carry skipples). Since the ordering of the trucks is not constrained in answering the problem, thus we can express it by stating truck 4 until 8 can't carry skipples, which is 
 \[ \bigwedge_{i=4,\cdots 8} c_i=0\]
 
 The second constraint is no two pallets of nuzzles may be in the same truck. This means that each trucks can only carry a maximum of 1 nuzzle. This can be represented by this formula
 \[ \bigwedge_{i=1,\cdots 8} a_i\leq1\]
 
 Last but not least, for every variables, it has to be greater than 0 because the minuimum number of a pallet in the truck is 0.
 \[ (\bigwedge_{i=1,\cdots 8} a_i\geq0) \wedge (\bigwedge_{i=1,\cdots 8} b_i\geq0) \wedge (\bigwedge_{i=1,\cdots 8} c_i\geq0) \wedge  (\bigwedge_{i=1,\cdots 8} d_i\geq0) \wedge (\bigwedge_{i=1,\cdots 8} e_i\geq0)\]
 
 Finally, we can solve the maximum number of $p_b$ by trying the biggest integer numbers from the term {\tt (= (+ b1 b2 b3 b4 b5 b6 b7 b8) $p_b$) } that yields sat results. We know the upper limit of $p_b$ candidate. If each of the 8 trucks can carry at most 8 pallets, there has to be at most 64 pallets. Thus, the upper limit of the candidate is $64-p_a-p_c-p_d-p_e=22$. We try $p_b=22$ but the yices-smt solver returns an unsat result. But, when we try $p_b=21$, the result is sat. Thus, the answer for the first question is 21. Here is the code to solve the first question: (full code can be seen at: TODO)
 
 {\footnotesize

{\tt  (benchmark test.smt }

{\tt  :logic QF\_UFLIA }

{\tt  :extrafuns }

{\tt  ((a1 Int) (a2 Int) (a3 Int) (a4 Int) (a5 Int) (a6 Int) (a7 Int) (a8 Int) }

{\tt  (b1 Int) (b2 Int) (b3 Int) (b4 Int) (b5 Int) (b6 Int) (b7 Int) (b8 Int) }

{\tt  (c1 Int) (c2 Int) (c3 Int) (c4 Int) (c5 Int) (c6 Int) (c7 Int) (c8 Int) }

{\tt  (d1 Int) (d2 Int) (d3 Int) (d4 Int) (d5 Int) (d6 Int) (d7 Int) (d8 Int) }

{\tt  (e1 Int) (e2 Int) (e3 Int) (e4 Int) (e5 Int) (e6 Int) (e7 Int) (e8 Int) }

{\tt  ) }

{\tt  :formula }

{\tt  ( and  }

{\tt  (<= (+ (* 800 a1) (* 1100 b1) (* 1000 c1) (* 2500 d1) (* 200 e1)) 8000) }

{\tt  (<= (+ (* 800 a2) (* 1100 b2) (* 1000 c2) (* 2500 d2) (* 200 e2)) 8000) }

{\tt  (<= (+ (* 800 a3) (* 1100 b3) (* 1000 c3) (* 2500 d3) (* 200 e3)) 8000) }

{\tt  (<= (+ (* 800 a4) (* 1100 b4) (* 1000 c4) (* 2500 d4) (* 200 e4)) 8000) }

{\tt  (<= (+ (* 800 a5) (* 1100 b5) (* 1000 c5) (* 2500 d5) (* 200 e5)) 8000) }

{\tt  (<= (+ (* 800 a6) (* 1100 b6) (* 1000 c6) (* 2500 d6) (* 200 e6)) 8000) }

{\tt  (<= (+ (* 800 a7) (* 1100 b7) (* 1000 c7) (* 2500 d7) (* 200 e7)) 8000) }

{\tt  (<= (+ (* 800 a8) (* 1100 b8) (* 1000 c8) (* 2500 d8) (* 200 e8)) 8000) }

{\tt  (<= (+ a1 b1 c1 d1 e1) 8) }

{\tt  (<= (+ a2 b2 c2 d2 e2) 8) }

{\tt  (<= (+ a3 b3 c3 d3 e3) 8) }

{\tt  (<= (+ a4 b4 c4 d4 e4) 8) }

{\tt  (<= (+ a5 b5 c5 d5 e5) 8) }

{\tt  (<= (+ a6 b6 c6 d6 e6) 8) }

{\tt  (<= (+ a7 b7 c7 d7 e7) 8) }

{\tt  (<= (+ a8 b8 c8 d8 e8) 8) }

{\tt  (<= (+ (* 800 a2) (* 1100 b2) (* 1000 c2) (* 2500 d2) (* 200 e2)) 8000) }

{\tt  (<= (+ (* 800 a3) (* 1100 b3) (* 1000 c3) (* 2500 d3) (* 200 e3)) 8000) }

{\tt  (<= (+ (* 800 a4) (* 1100 b4) (* 1000 c4) (* 2500 d4) (* 200 e4)) 8000) }

{\tt  (<= (+ (* 800 a5) (* 1100 b5) (* 1000 c5) (* 2500 d5) (* 200 e5)) 8000) }

{\tt  (<= (+ (* 800 a6) (* 1100 b6) (* 1000 c6) (* 2500 d6) (* 200 e6)) 8000) }

{\tt  (<= (+ (* 800 a7) (* 1100 b7) (* 1000 c7) (* 2500 d7) (* 200 e7)) 8000) }

{\tt  (<= (+ (* 800 a8) (* 1100 b8) (* 1000 c8) (* 2500 d8) (* 200 e8)) 8000) }

{\tt  (= (+ a1 a2 a3 a4 a5 a6 a7 a8) 4) }

{\tt  ;main!!!! }

{\tt  (= (+ b1 b2 b3 b4 b5 b6 b7 b8) 21) }

{\tt  (= (+ c1 c2 c3 c4 c5 c6 c7 c8) 8) }

{\tt  (= (+ d1 d2 d3 d4 d5 d6 d7 d8) 10) }

{\tt  (= (+ e1 e2 e3 e4 e5 e6 e7 e8) 20) }

{\tt  (>= a1 0) }

{\tt  (>= a2 0) }

{\tt  (>= a3 0) }

{\tt  (>= a4 0) }

{\tt  (>= a5 0) }

{\tt  (>= a6 0) }

{\tt  (>= a7 0) }

{\tt  (>= a8 0) }

{\tt  (>= b1 0) }

{\tt  (>= b2 0) }

{\tt  (>= b3 0) }

{\tt  (>= b4 0) }

{\tt  (>= b5 0) }

{\tt  (>= b6 0) }

{\tt  (>= b7 0) }

{\tt  (>= b8 0) }

{\tt  (>= c1 0) }

{\tt  (>= c2 0) }

{\tt  (>= c3 0) }

{\tt  (>= c4 0) }

{\tt  (>= c5 0) }

{\tt  (>= c6 0) }

{\tt  (>= c7 0) }

{\tt  (>= c8 0) }

{\tt  (>= d1 0) }

{\tt  (>= d2 0) }

{\tt  (>= d3 0) }

{\tt  (>= d4 0) }

{\tt  (>= d5 0) }

{\tt  (>= d6 0) }

{\tt  (>= d7 0) }

{\tt  (>= d8 0) }

{\tt  (>= e1 0) }

{\tt  (>= e2 0) }

{\tt  (>= e3 0) }

{\tt  (>= e4 0) }

{\tt  (>= e5 0) }

{\tt  (>= e6 0) }

{\tt  (>= e7 0) }

{\tt  (>= e8 0) }

{\tt  ;only three of the eight trucks have facility for cooling skipples, assume t1 2 3 }

{\tt  (= c4 0) }

{\tt  (= c5 0) }

{\tt  (= c6 0) }

{\tt  (= c7 0) }

{\tt  (= c8 0) }

{\tt  ;no two pallets of nuzzles may be in the same truck }

{\tt  (<= a1 1) }

{\tt  (<= a2 1) }

{\tt  (<= a3 1) }

{\tt  (<= a4 1) }

{\tt  (<= a5 1) }

{\tt  (<= a6 1) }

{\tt  (<= a7 1) }

{\tt  (<= a8 1) }

{\tt  ) }

{\tt   }

{\tt   }

{\tt  ) }

}

Applying {\tt yices-smt -m no1a.smt} yields the following result
within a fraction of a second: 
{ \footnotesize
{\tt sat }

{\tt  }

{\tt (= c6 0) }

{\tt (= e8 4) }

{\tt (= a4 1) }

{\tt (= a6 0) }

{\tt (= a5 1) }

{\tt (= c3 1) }

{\tt (= b5 1) }

{\tt (= c5 0) }

{\tt (= e1 0) }

{\tt (= c7 0) }

{\tt (= e4 4) }

{\tt (= b2 3) }

{\tt (= b4 1) }

{\tt (= c2 2) }

{\tt (= d5 2) }

{\tt (= d1 0) }

{\tt (= d2 1) }

{\tt (= e6 1) }

{\tt (= e2 1) }

{\tt (= e3 2) }

{\tt (= a1 1) }

{\tt (= b6 7) }

{\tt (= c4 0) }

{\tt (= c8 0) }

{\tt (= b7 2) }

{\tt (= e5 4) }

{\tt (= a8 0) }

{\tt (= d6 0) }

{\tt (= d7 2) }

{\tt (= a3 1) }

{\tt (= d3 1) }

{\tt (= d4 2) }

{\tt (= e7 4) }

{\tt (= b8 2) }

{\tt (= d8 2) }

{\tt (= c1 5) }

{\tt (= a2 0) }

{\tt (= a7 0) }

{\tt (= b1 2) }

{\tt (= b3 3) }

{\tt  }
}
Expressed in a picture this yields: TODO

In answering second question, there is an additional constraint that we have to implement. Prittles and crottles are an explosive combination: they are not allowed to be put in the same truck. This means that in a truck, the amount of prittles or crottles has to be zero (or both). Which means

\[ \bigwedge_{i=1,\cdots 8}  (b_i=0 \vee d_i=0)  \]

Implementing the second part is just adding some additional line to the first part as follows

{ \footnotesize
{\tt ;prittles and crottles are an explosive combination: they are not allowed to be put in the same truck. }

{\tt (or (= b1 0) (= d1 0)) }

{\tt (or (= b2 0) (= d2 0)) }

{\tt (or (= b3 0) (= d3 0)) }

{\tt (or (= b4 0) (= d4 0)) }

{\tt (or (= b5 0) (= d5 0)) }

{\tt (or (= b6 0) (= d6 0)) }

{\tt (or (= b7 0) (= d7 0)) }

{\tt (or (= b8 0) (= d8 0)) }

}
Again, we have to find the biggest value of $p_b$ using the same rule as the first part, starting from $p_b=22$. The biggest $p_b$ that we yielding sat result is 19. The output of {\tt yices-smt -m no1a.smt} yields the following result within a fraction of a second


{\tt sat }

{\tt  }

{\tt (= c6 0) }

{\tt (= e8 1) }

{\tt (= a4 0) }

{\tt (= a6 1) }

{\tt (= a5 0) }

{\tt (= c3 2) }

{\tt (= b5 0) }

{\tt (= c5 0) }

{\tt (= e1 4) }

{\tt (= c7 0) }

{\tt (= e4 1) }

{\tt (= b2 0) }

{\tt (= b4 7) }

{\tt (= c2 5) }

{\tt (= d5 3) }

{\tt (= d1 2) }

{\tt (= d2 1) }

{\tt (= e6 1) }

{\tt (= e2 2) }

{\tt (= e3 4) }

{\tt (= a1 1) }

{\tt (= b6 6) }

{\tt (= c4 0) }

{\tt (= c8 0) }

{\tt (= b7 0) }

{\tt (= e5 2) }

{\tt (= a8 1) }

{\tt (= d6 0) }

{\tt (= d7 2) }

{\tt (= a3 0) }

{\tt (= d3 2) }

{\tt (= d4 0) }

{\tt (= e7 5) }

{\tt (= b8 6) }

{\tt (= d8 0) }

{\tt (= c1 1) }

{\tt (= a2 0) }

{\tt (= a7 1) }

{\tt (= b1 0) }

{\tt (= b3 0) }

{\tt  }

Expressed in a picture this yields: TODO


\vspace{3mm}

%\includegraphics[height=3in,width=3in,angle=0]{eightq.jpg}

\vspace{3mm}

We check that indeed there are eight queens for which no two are
on the same row, column or diagonal.

\vspace{3mm}

{\bf Remark:} 

Our formula contains some redundancy: the requirement that every
row contains exactly one queen implies that there are exactly $n$
queens in total. By expressing that every column contains at least
one queen, from this property one concludes that also every column
contains at most one queen. We chose for this redundancy for
keeping the symmetry in the solution, and also following the
general strategy that introducing redundancy is often good for
efficiency.

\vspace{3mm}

{\bf Generalization:} 

As we generalized our approach for $n$ rather than 8, it is
interesting to see what happens for other values of $n$. For $n
> 10$ we have to take care of the notation: if we keep the
notation then it is not clear whether {\tt p111} represents 
$p_{1,11}$ or $p_{11,1}$. This is solved by putting an extra 
symbol between the two numbers. 

For $n=3$ the resulting formula is unsatisfiable, showing that
there is no solution. For $n = 4,5,6,\ldots$ the formula is
satisfiable, by which is a solution of the problem is found.
Efficiency is not a problem: for $n = 60$ there are 3600
variables and the formula consists of over 350,000 clauses, but 
still {\tt yices} succeeds in finding a solution within a few
seconds.

\subsection*{Problem: Designing chip}
Give a chip design containing two power components and ten
regular components satisfying the constraints. What if this last distance requirement of 18 is increased to 20? And what if it increased to 22? 

\subsection*{Solution:}
Let's say we the wide of the chip is $w$ and its height is $h$. We have $k$ normal component $n_1, n_2, \cdots n_k$ and $l$ power component $p_1, p_2, \cdots p_l$. 

\subsection*{Problem: Jobs scheduling}
Find a solution of this scheduling problem for which the total running time is minimal

\subsection*{Solution:}
We introduce 12 variables for each job: $t_1, t_2, \cdots , t_{12}$ representing the starting time of job $1, 2, \cdots, 12$ respectively. Each of the job has duration $d_1, d_2, ... d_{12}$. Most of the constrain in this problem is in form "Job $x$ may only start if jobs $y_1$ and $y_2$ and ... $y_n$ have been finished." This means that the starting time of job $x$ is always greater or equal than the time job $y_1, y_2, \cdots y_n$ finishes which is $t_{y_1}+d_{y_1}, t_{y_2}+d_{y_2}, \cdots t_{y_n}+d_{y_n}$ respectively. This statement can be formulated as follows:
\[ \bigwedge_{i=1,\cdots n} t_x \geq t_{y_i}+d_{y_i} \]

In this problem, we have $d_1, d_2, \cdots d_{12}$ equal to $6, 7, \cdots 12$ respectively. For each of the x we construct like above (TODO)

Another form of the constraint is "Job x may not start earlier than job $y_1, y_2, \cdots y_n$." It means that the starting time of job $x$ has to be greater than or equal to the starting time of job $y_1, y_2, \cdots y_n$. This statement can be formulated as follows:
\[ \bigwedge_{i=1,\cdots n} t_x \geq t_{y_i} \]

The last form of the constraint is "Jobs $y_1, y_2, \cdots y_n$ require a special equipment of which only one copy is available, so no two of these jobs may run at the same time". Which means: for every pair of job $(a,b) \in \{(y_1, y_2), (y_1, y_3), \cdots (y_{n-1}, y_{n})\}$ Job a has to be finished before job b started or job a has to be started after job b finished. This statement can be formulated as follows:

Let $Y=\{(y_1, y_2), (y_1, y_3), \cdots (y_{n-1}, y_{n})\}$

   {\Large $\forall_{(a,b) \in Y}$} $((t_a + d_a \leq t_b) \vee (t_a \geq t_b+d_b) )$
   \[ \forall_{a,b}\bigwedge_{i=1,\cdots n} t_x \geq t_{y_i} \]

Then, we have to make sure for every $t_1, t_2, \cdots t_12$ has to be greater than 0 because our starting time is $t=0$. For having the smallest total job time, we have to get the biggest starting job time from the smt-solver and try to change that number into smaller number until it reaches the smallest number that still yields sat result. Here is the initial code to solve the job scheduling problem: 
{\footnotesize

{\tt (benchmark test.smt }

{\tt :logic QF\_UFLRA }

{\tt :extrafuns }

{\tt ( (t1 Real) (t2 Real) (t3 Real) (t4 Real) (t5 Real) (t6 Real) (t7 Real) (t8 Real) (t9 Real) (t10 Real) (t11 Real) (t12 Real) }

{\tt ) }

{\tt :formula(and  }

{\tt (>= t1 0) }

{\tt (>= t2 0) }

{\tt (>= t3 0) }

{\tt (>= t4 0) }

{\tt (>= t5 0) }

{\tt (>= t6 0) }

{\tt (>= t7 0) }

{\tt (>= t8 0) }

{\tt (>= t9 0) }

{\tt (>= t10 0) }

{\tt (>= t11 0) }

{\tt (>= t12 0) }

{\tt (>= t3 (+ t1 6)) }

{\tt (>= t3 (+ t2 7)) }

{\tt (>= t5 (+ t3 8)) }

{\tt (>= t5 (+ t4 9)) }

{\tt (>= t7 (+ t3 8)) }

{\tt (>= t7 (+ t4 9)) }

{\tt (>= t7 (+ t6 11)) }

{\tt (>= t8 t5) }

{\tt (>= t9 (+ t5 10)) }

{\tt (>= t9 (+ t8 13)) }

{\tt (>= t11 (+ t10 15)) }

{\tt (>= t12 (+ t9 14)) }

{\tt (>= t12 (+ t11 16)) }

{\tt (or (<= (+ t5 10) t7) (>= t5 (+ t7 12))) }

{\tt (or (<= (+ t5 10) t10) (>= t5 (+ t10 15))) }

{\tt (or (<= (+ t7 12) t10) (>= t7 (+ t10 15))) }

{\tt  }

{\tt   }

{\tt  }

{\tt ) }

{\tt  }

{\tt ) }
	
}
From the initial result, we see that $t_{12}$ produces the biggest starting time, thus it is always likely the last job to run. We have $t_{12}=54$ from the initial result. We try to minimize $t_{12}$ by adding {\tt (= t12 x) } where x is the desired starting time. Based on our experiment, $x=42$ is the smallest number that still yields sat result. Thus, the minimal total job time is $42+d_{12}=42+17=59$. (Full final code can be seen at: TODO)

Applying {\tt yices-smt -m no3.smt} yields the following result
within a fraction of a second: 

{\footnotesize
{\tt sat }

{\tt  }

{\tt (= t8 15) }

{\tt (= t1 1) }

{\tt (= t4 0) }

{\tt (= t6 14) }

{\tt (= t3 7) }

{\tt (= t10 0) }

{\tt (= t5 15) }

{\tt (= t12 42) }

{\tt (= t2 0) }

{\tt (= t7 25) }

{\tt (= t9 28) }

{\tt (= t11 15) }

{\tt  }
}
Expressed in a picture this yields: TODO

\end{document}

